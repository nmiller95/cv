\begin{rubric}{Other Experience}
\subrubric{Research}
\entry*[2017]%
    \textbf{Summer research student}, Nicolaus Copernicus Astronomical Center of the Polish Academy of Sciences, Poland. Worked with Dr. Rodolfo Smiljanic over 4 weeks, using R to simulate a population of stars in the local neighbourhood and perform statistical analysis.
\entry*[2014 -- 2017]%
    \textbf{Participant}, International Astronomical Youth Camp (IAYC), Germany. Completed 4 projects: Python implementations of a radiative-convective atmospheric model and the habitable zone around different stars; remote observation and photometry of exoplanet hosts and variable stars with the 1.5m Leopold-Figl telescope, Austria.
    
%\subrubric{Observing}
%\entry*[Mar. 2020]
%    \textbf{5 nights}, with SOPHIE on T1.93 at Observatoire de Haute Provence, France
%\entry*[Sep. 2019]
%    \textbf{5 nights}, with SOPHIE on T1.93 at Observatoire de Haute Provence, France
    
\subrubric{Successful proposals}
\entry*[2020]
    \textit{High resolution and high signal to noise spectroscopy of the benchmark eclipsing binary system AI Phe}, ESO/UT2-Kueyen/UVES, 106.2138, 0.125~n, lead author: Paula Jofr\'{e} Pfeil.
\entry*[2020]
    \textit{Radial-velocity confirmation of circumbinary planet candidates}, ESO/HARPS, 106.212H, 29.6~n, lead author: Amaury Triaud.

\subrubric{Teaching \& Supervision}
\entry*[2018 -- $\cdots\cdot$]%
    \textbf{Lab demonstrator}, Keele Univeristy, United Kingdom. Teach in two undergraduate physics and astrophysics laboratories per semester. Gained certification in \emph{Introduction to Teaching and Demonstrating} in 2019.
\entry*[2017 -- $\cdots\cdot$ ]%
    \textbf{Leader}, International Astronomical Youth Camp (IAYC), Germany. Design and supervise astronomical projects for a group of 8-10 students during the 3-week summer camp. Assist with the selection process, planning logistics and running the information e-mail service. Led the online initiative `eIAYC 2020', which ran in place of the camp (cancelled by COVID-19).

\subrubric{Outreach}
\entry*[2019 -- $\cdots\cdot$ ]%
    \textbf{Organiser}, Pint of Science Stoke, United Kingdom. Select venues and coordinate speakers as part of a local team for this international public engagement event.
\entry*[2019 -- $\cdots\cdot$]%
    \textbf{Outreach volunteer}, Keele Observatory and with the Keele University Stardome. 
\entry*[2017 -- 2019]%
    \textbf{Supervisor}, ESO Astronomy Camp, Italy. Responsible for the safety and wellbeing of 55 students aged 16-18 for the duration of the camp. Shared knowledge, prompted discussion and took photographs of the events.
\entry*[2017 -- 2018]%
    \textbf{Outreach volunteer}, University of Warwick Department of Physics, United Kingdom. Took the planetarium to local schools, ran stargazing and Q\&A sessions.
\entry*[2016 -- 2018]%
    \textbf{Founder and President} of the University of Warwick Astronomy Society. Formed the society, recruited a committee, led weekly talks, workshops, observations plus academic support and international trips for 100+ members. The society won “Warwick SU Best New Society 2017”.
\end{rubric}