\begin{rubric}{Teaching \& Outreach}
\subrubric{Teaching Responsibilities}
\entry*[2018 -- 2022]%
    \textbf{Lab demonstrator}, Keele University, United Kingdom. Taught in two undergraduate physics and astrophysics laboratories or problems classes each semester. Gained certification in the training course \emph{Introduction to Teaching and Demonstrating} in 2019.

\subrubric{Outreach \& Public Engagement}
\entry*[2023 -- ~~~~~\,\,\,~~~]%$\cdots\cdot$]%
    \textbf{President}, International Workshop for Astronomy e.V. (IWA), Germany. IWA is the non-profit organisation responsible for overseeing the International Astronomical Youth Camp (IAYC), a 3-week astronomy research camp where 64 young people aged 16-24 from around the world work together in small groups on a project. I steer the present and future direction of IWA, supervise the team of $\sim$15 volunteers, and perform many essential functional tasks.
\entry*[2017 -- ~~~~~\,\,\,~~~]%-- $\cdots\cdot$ ]%
    \textbf{Leader}, International Astronomical Youth Camp (IAYC), Germany. Design and supervise astronomical research projects for a group of 8-10 students during the 3-week summer camp. Contribute to the selection process, logistics and day-to-day running of the camp.
\entry*[2021 -- 2023]%
    \textbf{Vice President}, International Workshop for Astronomy e.V. (IWA), Germany.
\entry*[2019 -- 2022]%
    \textbf{Outreach volunteer}, Keele Observatory and with the Keele University Stardome. 
\entry*[2019 -- 2020]%
    \textbf{Organiser}, Pint of Science Stoke, United Kingdom. Selected venues and coordinated speakers as part of a local team for this international public engagement event.
\entry*[2017 -- 2019]%
    \textbf{Supervisor}, ESO Astronomy Camp, Italy. Responsible for the safety and wellbeing of 55 students aged 16-18 for the duration of the camp.% Shared knowledge, prompted discussion and took photographs of the events.
\entry*[2017 -- 2018]%
    \textbf{Outreach volunteer}, University of Warwick Department of Physics, United Kingdom. Took the planetarium to local schools, ran stargazing and Q\&A sessions.

    
\end{rubric}